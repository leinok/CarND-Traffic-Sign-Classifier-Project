%%%%%%%%%%%%%%%%%%%%%%%%%%%%%%%%%%%%%%%%%%%%%%%%%%%%%%%%%%%%%%%%%%%%%%
% LaTeX Example: Project Report
%
% Source: http://www.howtotex.com
%
% Feel free to distribute this example, but please keep the referral
% to howtotex.com
% Date: March 2011 
% 
%%%%%%%%%%%%%%%%%%%%%%%%%%%%%%%%%%%%%%%%%%%%%%%%%%%%%%%%%%%%%%%%%%%%%%
% How to use writeLaTeX: 
%
% You edit the source code here on the left, and the preview on the
% right shows you the result within a few seconds.
%
% Bookmark this page and share the URL with your co-authors. They can
% edit at the same time!
%
% You can upload figures, bibliographies, custom classes and
% styles using the files menu.
%
% If you're new to LaTeX, the wikibook is a great place to start:
% http://en.wikibooks.org/wiki/LaTeX
%
%%%%%%%%%%%%%%%%%%%%%%%%%%%%%%%%%%%%%%%%%%%%%%%%%%%%%%%%%%%%%%%%%%%%%%
% Edit the title below to update the display in My Documents
%\title{Project Report}
%
%%% Preamble
\documentclass[paper=a4, fontsize=11pt]{scrartcl}
\usepackage[T1]{fontenc}
\usepackage{fourier}

\usepackage[english]{babel}															% English language/hyphenation
\usepackage[protrusion=true,expansion=true]{microtype}	
\usepackage{amsmath,amsfonts,amsthm} % Math packages
\usepackage[pdftex]{graphicx}	
\usepackage{url}
\usepackage{subcaption}
%%% Custom sectioning
\usepackage{sectsty}
\allsectionsfont{\centering \normalfont\scshape}


%%% Custom headers/footers (fancyhdr package)
\usepackage{fancyhdr}
\pagestyle{fancyplain}
\fancyhead{}											% No page header
\fancyfoot[L]{}											% Empty 
\fancyfoot[C]{}											% Empty
\fancyfoot[R]{\thepage}									% Pagenumbering
\renewcommand{\headrulewidth}{0pt}			% Remove header underlines
\renewcommand{\footrulewidth}{0pt}				% Remove footer underlines
\setlength{\headheight}{13.6pt}


%%% Equation and float numbering
\numberwithin{equation}{section}		% Equationnumbering: section.eq#
\numberwithin{figure}{section}			% Figurenumbering: section.fig#
\numberwithin{table}{section}				% Tablenumbering: section.tab#


%%% Maketitle metadata
\newcommand{\horrule}[1]{\rule{\linewidth}{#1}} 	% Horizontal rule

\title{
		%\vspace{-1in} 	
		\usefont{OT1}{bch}{b}{n}
		\normalfont \normalsize \textsc{Udacity Autonomous Driving Car} \\ [25pt]
		\horrule{0.5pt} \\[0.4cm]
		\huge Project 2 Traffic sign classifier \\
		\horrule{2pt} \\[0.5cm]
}
\author{
		\normalfont 								\normalsize
        \today
}
\date{}


%%% Begin document
\begin{document}
\maketitle
\section{Summary of Project 2}
Project 2 and corresponding course talked about use Tensorflow to implement traffic signs classification use LeNet architecture. From the project, I have the folloing thoughts:
\begin{enumerate}
	\item Data preprocessing matters. I tried directly using the original image, using min-max scaling and zero-score scaling. The results can be significant different. Directly using original training data achieved less than $80\%$ training accuracy. However, using zero-score I achieved $99\%$ training accuracy and $93\%$ testing accuracy. The improvement is significant.
	\item Learning rate is important. Large learning rate converge fast but results in lower accuracy, lower learning rate converge slower. The rule of thumb of this value can be 0.001.
	\item GPU is much faster than CPU. I also tried training CPU in my machine, the GPU-based training and testings can be speed up by $50X$.
	\item Dropout is very helpful. The testing accuracy improved from $90\%$ to $93\%$.
\end{enumerate}

\section{Possible improvements}
There is no difference between deep learning and more conventional "neural network". I can try deeper layers (with more than 10 convolutional layers) or new architecture introduced in recent years such as ResNet to improve the classification rate.

More detail can be found at ipython file.
\end{document}